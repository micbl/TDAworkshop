\documentclass{beamer}
	\usefonttheme{default}     % Font theme: serif
	\usecolortheme[snowy]{owl}

\usepackage{parskip}
	\setlength{\parskip}{\smallskipamount} 


\title{
%\includegraphics[scale = 0.18]{Figs/Topological-Data-Illustration.jpg}\\
\vspace{0.3cm}
\Large{Topological Data Analysis in Python}}
\date{26\textsuperscript{th} - 28\textsuperscript{th} of October 2020}
\author[Michael Bleher]{organized by: \\  Michael Bleher, Maximilian Schmahl and Daniel Spitz}
\institute{Heidelberg University}

\begin{document}
\titlepage

\AtBeginSection[]
{
	\begin{frame}{Contents}
	\setcounter{tocdepth}{1}
	\tableofcontents[currentsection, ]
	\end{frame}
}

\AtBeginSubsection[] {
    \begin{frame}<beamer>
    \frametitle{Inhalt} %
		\setcounter{tocdepth}{2}
    \tableofcontents[currentsubsection]
    \end{frame}
}


\section*{Contents}
\begin{frame}[plain]{Contents}
\setcounter{tocdepth}{1}
\tableofcontents
\end{frame}


\section{Programme}
\begin{frame}{Programme}

\end{frame}


\section{Topology in Data Analysis}
\begin{frame}{Topology}
Topology is the field of mathematics that studies shapes.

Topology is ``blind'' to continuous deformations.

(obligatory pictures: interval, circle, sphere, punctured sphere, torus, coffee cup)
\end{frame}

\begin{frame}{Topology}

\end{frame}

\begin{frame}{Topology}

\end{frame}

\begin{frame}{Topology}

\end{frame}



\section{Overview scikit-tda}
\begin{frame}{Overview scikit-tda}

Libraries:
\begin{itemize}
	\item Ripser
	\item Kepler Mapper
	\item Persim
	\item CechMate
	\item TaDAsets
\end{itemize}

\vfill
see: \url{https://scikit-tda.org/}
\end{frame}






\section{The Mapper Algorithm}

\begin{frame}

\end{frame}




\end{document}

